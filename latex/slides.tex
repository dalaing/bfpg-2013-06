\documentclass{beamer}

\usepackage{amsmath}
\usepackage{fancyvrb}

\usetheme{Berlin}

\AtBeginSection[]%
{%
\begin{frame}%
  \begin{center}%
    \usebeamerfont{section title}\insertsection%
  \end{center}%
\end{frame}%
}%

\setbeamersize{text margin left=15pt}
\newcommand{\ca}[1]{{\color{blue}#1}}
\newcommand{\cb}[1]{{\color{violet}#1}}
\newcommand{\cc}[1]{{\color{red}#1}}
\newcommand{\cd}[1]{{\color{orange!90!black}#1}}
\newcommand{\ce}[1]{{\color{green!50!black}#1}}

\mode<presentation>

\title{Laws for Folds and Theorems for Free}

\author{Dave Laing}

\begin{document}

\begin{frame}
    \titlepage
\end{frame}

\begin{frame}
    \frametitle{Outline}
    \tableofcontents[pausesections]
\end{frame}

\section{Preliminaries}

\begin{frame}[fragile]
    \frametitle{Bottom}
\begin{itemize}
    \item<1-> We normally think of data types like this:\\
    \verb?  data Bool = False?\\
    \verb?            | True?\\
    \item<2-> But in Haskell they're actually like this:\\
    \verb?  data Bool = False?\\ 
    \verb?            | True?\\ 
    \verb?            |? $\bot$
\end{itemize}

\end{frame}

\begin{frame}[fragile]
    \frametitle{Strict and Non-Strict Functions}

\begin{itemize}
    \item A function $f$ is strict if \verb?f ?$\bot$\verb? = ?$\bot$
    \item Otherwise $f$ is non-strict
    \item For example if \\
    \verb?  f = const 2?\\ 
    then \\
    \verb?  f ?$\bot$\verb? = 2?
\end{itemize}

\end{frame}

\begin{frame}[fragile]
    \frametitle{Non-strict example}

\begin{itemize}
    \item Consider \\
    \verb?  and :: Bool -> Bool -> Bool?\\
    \verb?  and False _  = False?\\
    \verb?  and True  x  = x?\\
    \verb?  and _     _  = ?$\bot$\\
    \item Strict in the first argument
    \begin{itemize}
        \item \verb?and ?$\bot$\verb? x = ?$\bot$
    \end{itemize}
    \item Non-strict in the second argument
    \begin{itemize}
        \item \verb?and False ?$\bot$\verb? = False?
        \item \verb?and True  ?$\bot$\verb? = ?$\bot$
    \end{itemize}
\end{itemize}

\end{frame}

\begin{frame}[fragile]
    \frametitle{Strict example}

\begin{itemize}
    \item Consider \\
    \verb?  and' :: Bool -> Bool -> Bool?\\
    \verb?  and' False False = False?\\
    \verb?  and' False True  = False?\\
    \verb?  and' True  False = False?\\
    \verb?  and' True  True  = True?\\
    \item Strict in the first argument
    \begin{itemize}
        \item\verb?and' ?$\bot$\verb? x = ?$\bot$
    \end{itemize}

    \item Strict in the second argument
    \begin{itemize}
        \item\verb?and' x ?$\bot$\verb? = ?$\bot$
    \end{itemize}
\end{itemize}

\end{frame}

\begin{frame}[fragile]
    \frametitle{Partial data}
    \begin{itemize}
        \item $(\bot, \bot) \ne \bot$
        \begin{itemize}
            \item A pair on the left, a... something... on the right
        \end{itemize}
        \item $[\bot] \ne \bot$
        \begin{itemize}
            \item It's a list with... something... in it
        \end{itemize}
        \item $(1 : 2 : 3 : \bot)$ is a list with at least three elements in it
        \item Infinite lists are partial lists
    \end{itemize}
\end{frame}

\begin{frame}[fragile]
    \frametitle{Proofs with Partial Lists}
    \begin{itemize}
        \item Prove $P($\verb?[]?$)$ and $P(xs) \rightarrow P(x:xs)$ 
            \begin{itemize}
                \item $P$ holds for all finite lists
            \end{itemize}
        \item Prove $P(\bot)$ and $P(xs) \rightarrow P(x:xs)$ 
            \begin{itemize}
                \item $P$ holds for all partial lists
            \end{itemize}
        \item Prove $P(\bot)$, $P($\verb?[]?$)$ and $P(xs) \rightarrow P(x:xs)$ 
            \begin{itemize}
                \item $P$ holds for all lists
            \end{itemize}
    \end{itemize}
\end{frame}

\section{Fold fusion}

\begin{frame}[fragile]
    \frametitle{Fold Fusion}

    \begin{itemize}
        \item When does \\
        \verb?  f . foldr g a = foldr h b? ?
    \end{itemize}
\end{frame}

\begin{frame}[fragile]
    \frametitle{Deriving the Fold Fusion Law}

\begin{itemize}
    \item<1-> Case $\bot$    
    \item<2-> \verb?f (foldr g a ?$\bot$\verb?) = foldr h b ?$\bot$    
    \item<3-> \verb?f ?$\bot$\verb? = ?$\bot$
\end{itemize}

\end{frame}

\begin{frame}[fragile]
    \frametitle{Deriving the Fold Fusion Law}

\begin{itemize}
    \item<1-> Case \verb?[]?    
    \item<2-> \verb?f (foldr g a []) = foldr h b []?
    \item<3-> \verb?f a = b?
\end{itemize}

\end{frame}

\begin{frame}[fragile]
    \frametitle{Deriving the Fold Fusion Law}

\begin{itemize}
    \item<1-> Case \verb?(x:xs)? 
    \item<2-> Assume \verb?f (foldr g a xs) = foldr h b xs?
    \item<3-> \verb?f (foldr g a (x:xs)) = foldr h b (x:xs)?
    \item<4-> \verb?f (g x (foldr g a xs)) = h x (foldr h b xs)?
    \item<5-> Let \\
    \verb?  y = foldr g a xs?
    \item<6-> Note that  \\
    \verb?  f(y) = f (foldr g a xs) = foldr h b xs?
    \item<7-> And so \\
    \verb?  f (g x y) = h x (f y)?
\end{itemize}

\end{frame}

\begin{frame}[fragile]
    \frametitle{The Fold Fusion Law}

So the conditions are
\begin{itemize}
    \item \verb?f? is strict\\
    \item \verb?f a = b?\\
    \item \verb?f (g x y) = h x (f y)?\\
\end{itemize}

\end{frame}

\begin{frame}
    \frametitle{Uses}

\begin{itemize}
    \item Reason about things that already exist
    \item Speed up your code
    \begin{itemize}
        \item directly
        \item via rules pragmas
        \item let the compiler do it (but know it can happen)
    \end{itemize}
\end{itemize}

\end{frame}

\begin{frame}[fragile]
    \frametitle{Fold-Map Fusion}

\begin{itemize}
    \item Remember that \\
    \verb?  map g = foldr ((:) . g) []?\\
    so \\
    \verb?  foldr f a . map g = foldr h b?\\
    is subject to fold fusion.
    \item Fold fusion conditions imply that
    \begin{itemize}
        \item \verb?a = b?
        \item \verb?f . g = h?
    \end{itemize}
    \item Resulting law is \\
    \verb?  foldr f a . map g = foldr (f . g) a?
\end{itemize}

\end{frame}

\begin{frame}[fragile]
    \frametitle{Fold-Concat Fusion}

\begin{itemize}
    \item Remember that\\
    \verb?  concat = foldr (++) []?\\
    so\\
    \verb?  foldr f a . concat?\\
    is subject to fold fusion.\\
    \item Spolier alert:\\
    \verb?  foldr f a . concat =  foldr (flip (foldr f)) a?
\end{itemize}

\end{frame}

\begin{frame}[fragile]
    \frametitle{Bookkeeping law}

    \begin{itemize}
        \item Fold fusion works with map and with concat.
        \item So \\
        \verb?  foldr f a . concat = foldr g b . map h?
        \item Works out as \\
        \verb?  foldr f a . concat = foldr f a . map (foldr f a)?
        \item Examples:
        \begin{itemize}
            \item \verb?sum . concat = sum . map sum?
            \item \verb?concat . concat = concat . map concat?
        \end{itemize}
    \end{itemize}

\end{frame}

\begin{frame}[fragile]
    \frametitle{Scans}

\begin{itemize}
    \item Like a fold that shows its working
    \item Uses \verb?inits?
        \begin{itemize}
            \item \verb?inits [1,2,3] = [[],[1],[1,2],[1,2,3]]?
        \end{itemize}
    \item \verb?scanl f a = map (foldl f a) . inits?
    \item For example
        \begin{itemize}
            \item \verb?scanl (+) 0 [1..4] = [0, 1, 3, 6, 10]?
        \end{itemize}
    \item Can work from the right as well
        \begin{itemize}
            \item \verb?scanr f a = map (foldr f a) . tails?
        \end{itemize}
\end{itemize}

\end{frame}

\begin{frame}[fragile]
    \frametitle{Fold-Scan Fusion}

\begin{itemize}
    \item Consider\\ 
    \verb?  ?$1 + x_0 + x_0 * x_1 + x_0 * x_1 * x_2 + \ldots$\\
    \item Group terms\\
    \verb?  ?$1 + x_0 * (1 + x_1 * (1 + x_2 * \ldots))$\\
    \item Haskellizes to\\
    \verb?  foldr1 (+) . scanl (*) 1?\\
\end{itemize}

\end{frame}

\begin{frame}[fragile]
    \frametitle{Fold-Scan Fusion}

\begin{itemize}
    \item When does\\
        \verb?  foldr1 (+) . scanl (*) e = foldr (?$\odot$\verb?) e?\\
    \item If
        \begin{itemize}
            \item \verb?*? is associative with unit \verb?e?
            \item \verb?*? distributes over \verb?+?
        \end{itemize}
    \item Then \\
        \verb?foldr1 (+) . scanl (*) e = foldr (?$\odot$\verb?) e?\\
        \verb?  where ? \\
        \verb?    x ?$\odot$\verb? y = e + (x * y)?
\end{itemize}

\end{frame}

\begin{frame}[fragile]
    \frametitle{Maximum Subset Sum}

\begin{itemize}
    \item Problem posed and solved by Richard Bird
    \item Example of behaviour-preserving transformations from a trivially correct program to an efficient program
    \item Goal is to find the maximum sum of contiguous elements
    \item
    \verb?mss = maxlist . map sum . segs? \\
    \verb?  where? \\
    \verb?    maxlist = foldr1 max? \\
    \verb?    segs = concat . map inits . tails?
\end{itemize}

\end{frame}

\begin{frame}[fragile,t]
    \frametitle{Maximum Subset Sum}
    
For example: \verb?mss [1, 2, -3]?

\begin{center}
\begin{overprint}

\onslide<2>
\begin{semiverbatim}
                       \cb{tails} [1, 2,-3] = 
  [                  [1, 2,-3]  , 
              [2,-3]  , 
         [-3]  , 
      []   ]
\end{semiverbatim}

\onslide<3>
\begin{semiverbatim}
         \cb{map inits . tails} \$ [1, 2,-3] = 
  [                  [1, 2,-3]  , 
              [2,-3]  , 
         [-3]  , 
      []   ]
\end{semiverbatim}

\onslide<4>
\begin{semiverbatim}
         \cb{map inits . tails} \$ [1, 2,-3] = 
  [ [ [],[ 1],[1, 2],[1, 2,-3] ], 
    [ [],[ 2],[2,-3] ], 
    [ [],[-3] ], 
    [ [] ] ]
\end{semiverbatim}

\onslide<5>
\begin{semiverbatim}
\cb{concat . map inits . tails} \$ [1, 2,-3] = 
  [ [ [],[ 1],[1, 2],[1, 2,-3] ], 
    [ [],[ 2],[2,-3] ], 
    [ [],[-3] ], 
    [ [] ] ]
\end{semiverbatim}

\onslide<6>
\begin{semiverbatim}
\cb{concat . map inits . tails} \$ [1, 2,-3] = 
  [   [],[ 1],[1, 2],[1, 2,-3]  , 
      [],[ 2],[2,-3]  , 
      [],[-3]  , 
      []   ]
\end{semiverbatim}

\onslide<7>
\begin{semiverbatim}
                        \cb{segs} [1, 2,-3] = 
  [   [],[ 1],[1, 2],[1, 2,-3]  , 
      [],[ 2],[2,-3]  , 
      [],[-3]  , 
      []   ]
\end{semiverbatim}

\onslide<8>
\begin{semiverbatim}
            \cb{map sum . segs} \$ [1, 2,-3] = 
  [   [],[ 1],[1, 2],[1, 2,-3]  , 
      [],[ 2],[2,-3]  , 
      [],[-3]  , 
      []   ]
\end{semiverbatim}

\onslide<9>
\begin{semiverbatim}
            \cb{map sum . segs} \$ [1, 2,-3] = 
  [    0,  1 ,    3 ,       0   , 
       0,  2 ,   -1   , 
       0, -3  , 
       0   ]
\end{semiverbatim}

\onslide<10>
\begin{semiverbatim}
  \cb{maxlist . map sum . segs} \$ [1, 2,-3] = 
  [    0,  1 ,    \ca{3} ,       0   , 
       0,  2 ,   -1   , 
       0, -3  , 
       0   ]
\end{semiverbatim}

\onslide<11>
\begin{semiverbatim}
  \cb{maxlist . map sum . segs} \$ [1, 2,-3] = 3
\end{semiverbatim}

\onslide<12>
\begin{semiverbatim}
                         \cb{mss} [1, 2,-3] = 3
\end{semiverbatim}

\end{overprint}
\end{center}

\vspace{10pt}

\begin{overprint}
    
\onslide<1>
\begin{semiverbatim}
mss = maxlist . map sum . segs
  where
    maxlist = foldr1 max
    segs = concat . map inits . tails
\end{semiverbatim}

\onslide<2>
\begin{semiverbatim}
mss = maxlist . map sum . segs
  where
    maxlist = foldr1 max
    segs = concat . map inits . \cb{tails}
\end{semiverbatim}

\onslide<3-4>
\begin{semiverbatim}
mss = maxlist . map sum . segs
  where
    maxlist = foldr1 max
    segs = concat . \cb{map inits . tails}
\end{semiverbatim}

\onslide<5-6>
\begin{semiverbatim}
mss = maxlist . map sum . segs
  where
    maxlist = foldr1 max
    segs = \cb{concat . map inits . tails}
\end{semiverbatim}

\onslide<7>
\begin{semiverbatim}
mss = maxlist . map sum . \cb{segs}
  where
    maxlist = foldr1 max
    segs = concat . map inits . tails
\end{semiverbatim}

\onslide<8-9>
\begin{semiverbatim}
mss = maxlist . \cb{map sum . segs}
  where
    maxlist = foldr1 max
    segs = concat . map inits . tails
\end{semiverbatim}

\onslide<10-11>
\begin{semiverbatim}
mss = \cb{maxlist . map sum . segs}
  where
    maxlist = foldr1 max
    segs = concat . map inits . tails
\end{semiverbatim}

\onslide<12>
\begin{semiverbatim}
\cb{mss} = maxlist . map sum . segs
  where
    maxlist = foldr1 max
    segs = concat . map inits . tails
\end{semiverbatim}

\end{overprint}

\end{frame}

\begin{frame}[fragile,t]
    \frametitle{Maximum Subset Sum}

\begin{center}
\begin{overprint}

\onslide<1>
\begin{semiverbatim}
  \cb{mss}
=    \{ definition of mss \}
  \cb{maxlist . map sum . segs}
\end{semiverbatim}

\onslide<2>
\begin{semiverbatim}
  maxlist . map sum . \cb{segs}
=    \{ definition of segs \}
  maxlist . map sum . \cb{concat . map inits . tails}
\end{semiverbatim}

\onslide<3>
\begin{semiverbatim}
  maxlist . \cb{map sum . concat} . map inits . tails
=    \{ bookkeepers law \}
  maxlist . \cb{concat . map (map sum)} . map inits . tails
\end{semiverbatim}

\onslide<4>
\begin{semiverbatim}
  \cb{maxlist . concat} . map (map sum) . map inits . tails
=    \{ bookkeepers law \}
  \cb{maxlist . map maxlist} . map (map sum) . map inits . tails
\end{semiverbatim}

\onslide<5>
\begin{semiverbatim}
  maxlist . \cb{map maxlist . map (map sum) . map inits} . tails
=    \{ functor law \}
  maxlist . \cb{map (maxlist . map sum . inits)} . tails
\end{semiverbatim}

\onslide<6>
\begin{semiverbatim}
  maxlist . map (maxlist . \cb{map sum . inits}) . tails
=    \{ definition of scanl \}
  maxlist . map (maxlist . \cb{scanl (+) 0}) . tails
\end{semiverbatim}

\onslide<7>
\begin{semiverbatim}
  maxlist . map (\cb{maxlist} . scanl (+) 0) . tails
=    \{ definition of maxlist \}
  maxlist . map (\cb{foldr1 max} . scanl (+) 0) . tails
\end{semiverbatim}

\onslide<8>
\begin{semiverbatim}
  maxlist . map (\cb{foldr1 max . scanl (+) 0}) . tails
=    \{ fold-scan fusion \}
     \{ + associative with unit 0 \}
     \{ + distributes over max \}
  maxlist . map (\cb{foldr addpos 0}) . tails
    where
      addpos x y = 0 `max` (x + y)
\end{semiverbatim}

\onslide<9>
\begin{semiverbatim}
  maxlist . \cb{map (foldr addpos 0) . tails}
=    \{ definition of scanr \}
  maxlist . \cb{scanr addpos 0}
\end{semiverbatim}

\end{overprint}
\end{center}

\end{frame}

\section{Natural transformations}

\begin{frame}[fragile]
    \frametitle{Definition}

    \begin{itemize}
        \item Take two functors \\
        \verb?  F, G?
        \item Add a function \\
        \verb?  f :: A -> B?
        \item The natural transformation $\eta$ makes \\ 
            $G(f) \circ \eta_A = \eta_B \circ F(f)$
    \end{itemize}

\end{frame}

\begin{frame}[fragile]
    \frametitle{What?}
\begin{itemize}
    \item Examples help
    \item $F$\verb? = [[]]?
        \begin{itemize}
            \item $F(f)$\verb? = map (map f)?
        \end{itemize}
    \item $G$\verb? = []?
        \begin{itemize}
            \item $G(f)$\verb? = map f?
        \end{itemize}
    \item That means $\eta$\verb? = concat? is a natural transformation between the list functor and the list-of-lists functor
        \begin{itemize}
            \item \verb?map f . concat = concat . map (map f)?
        \end{itemize}
    \item<2-> Bam!
        \begin{itemize}
            \item<3-> New law
        \end{itemize}
\end{itemize}
\end{frame}

\begin{frame}[fragile]
    \frametitle{More examples}
\begin{itemize}
    \item $F$\verb? = ?$G$\verb? = []?
        \begin{itemize}
            \item $F(f)$\verb? = ?$G(f)$\verb? = map f?
        \end{itemize}
    \item Lots of options for $\eta$
        \begin{itemize}
            \item \verb?map f . reverse = reverse . map f?
            \item \verb?map f . tail = tail . map f?
            \item \verb?map f . init = init . map f?
        \end{itemize}
    \item<2-> Bam! Bam! Bam!
        \begin{itemize}
            \item<3-> New laws aplenty
        \end{itemize}
    \item<4-> Any re-arrangement of the list will do
\end{itemize}
\end{frame}

\section{Parametericity}

\begin{frame}[fragile]
    \frametitle{Polymorphic functions}

\begin{itemize}
    \item Consider \\
        \verb?g :: forall A. [A] -> [A]?\\
    \item Could be \verb?reverse?
    \item Could be \verb?tail?
    \item Can only be a function that somehow rearranges the elements in the list
\end{itemize}

\end{frame}

\begin{frame}[fragile]
    \frametitle{A free theorem appears}

\begin{itemize}
    \item We know that \\
    \verb?  map f . reverse = reverse . map f?
    \item We also know that\\
    \verb?  map f . tail = tail . map f?
    \item Actually true that\\
    \verb?  map f . g = g . map f?\\
    whenever\\
    \verb?  g :: forall A. [A] -> [A]?\\
\end{itemize}

\end{frame}

\begin{frame}[fragile]
    \frametitle{Polymorphism required}
\begin{itemize}
    \item Consider\\
    \verb?  g :: [Int] -> [Int]?\\
    \verb?  g = filter odd?
    \item and\\
    \verb?  f :: Int -> Int?\\
    \verb?  f x = 2 * x?
    \item Pretty clear that\\
    \verb?  map f . g ?$\ne$\verb? g . map f?
    \item Stupid types, ruining stuff
\end{itemize}
\end{frame}

\begin{frame}[fragile]
    \frametitle{The laws are \textit{in} the types}
\begin{itemize}
    \item The laws can be derived mechanically
    \item Works for a variety of different typed functional languages
    \item Functions involved must be strict if \verb?fix? is in the language
\end{itemize}
\end{frame}

\begin{frame}[fragile]
    \frametitle{Relations on types}

\begin{itemize}
    \item The relationship between the sets $A$ and $\acute{A}$ is denoted by $\mathcal{A} : A \Leftrightarrow \acute{A}$
    \item Equivalent to $\mathcal{A} \subseteq A \times \acute{A}$
    \item So $(x, \acute{x}) \in \mathcal{A}$ means $x$ and $\acute{x}$ are related by $\mathcal{A}$ 
    \item If $t : T$ then $(t, t) \in \mathcal{T}$
    \item We can specialize the relation $\mathcal{A}$ to a function \verb?a : ?$A \rightarrow \acute{A}$
        \begin{itemize}
            \item Then $(x, \acute{x}) \in a$ is the same as \\
                $a\ x = \acute{x}$
        \end{itemize}
\end{itemize}

\end{frame}

\begin{frame}[fragile]
    \frametitle{Building relations - pairs}

\begin{itemize}
    \item 
        $((x,y),(\acute{x},\acute{y})) \in \mathcal{A} \times \mathcal{B}$ \\
        \verb?   ?is equivalent to \\
        $(x,\acute{x}) \in \mathcal{A}$ and $(y, \acute{y}) \in \mathcal{B}$
    \item If we specialize $\mathcal{A}$ and $\mathcal{B}$ to functions $a$ and $b$ then \\
        $(a \times b)(x, y) = (a\ x, b\ y)$
\end{itemize}

\end{frame}

\begin{frame}[fragile]
    \frametitle{Building relations - lists}

\begin{itemize}
    \item 
        $([x_0, \ldots, x_n],[\acute{x}_0, \ldots, \acute{x}_n]) \in [\mathcal{A}]$ \\
        \verb?    ?is equivalent to \\
        $(x_0, \acute{x}_0) \in \mathcal{A}, \ldots, (x_n, \acute{x}_n) \in \mathcal{A}$
    \item If we specialize $\mathcal{A}$ to function $a$ then \\
        $(xs,\acute{xs}) \in [a]$ \\
        \verb?    ?for all $i$, $a\ x_i = \acute{x}_i$ 
    \item Works out as \\
        $map\ a\ xs = \acute{xs}$
\end{itemize}

\end{frame}

\begin{frame}[fragile]
    \frametitle{Building relations - from other relations}

\begin{itemize}
    \item
        $(f, \acute{f}) \in \mathcal{A} \rightarrow \mathcal{B}$ \\
        \verb?    ?is equivalent to \\
        for all $(x, \acute{x}) \in \mathcal{A}$, $(f\ x, \acute{f}\ \acute{x}) \in \mathcal{B}$
    \item If we specialize $\mathcal{A}$ and $\mathcal{B}$ to functions $a$ and $b$ then \\
        $(f,\acute{f}) \in a \rightarrow b$ \\
        \verb?    ?is equivalent to \\
        $\acute{f} \circ a = b \circ f$
\end{itemize}

\end{frame}

\begin{frame}[fragile]
    \frametitle{Building relations - foralls}

\begin{itemize}
    \item 
        $(g,\acute{g}) \in \forall \mathcal{X}. \mathcal{F}(\mathcal{X})$ \\
        \verb?    ?is equivalent to\\
        for all $\mathcal{A} : A \Leftrightarrow \acute{A}$, $(g_A, \acute{g}_{\acute{A}}) \in \mathcal{F}(\mathcal{A})$
    \item Means that for a relation through a forall, we add a relation and impose no restrictions on it 
    \begin{itemize}
        \item since it has to work for all relations we could have used
    \end{itemize}
\end{itemize}

\end{frame}

\begin{frame}[fragile,t]
    \frametitle{Rearranging functions}

\begin{overprint}    

\onslide<1>    
$g : \forall W. [W] \rightarrow [W]$

\onslide<2>
$(g,g) \in \forall \mathcal{W}. [\mathcal{W}] \rightarrow [\mathcal{W}]$

\onslide<3>
for all $\mathcal{X} : X \Leftrightarrow \acute{X}$\\
\verb?  ?$(g_X, g_{\acute{X}}) \in [\mathcal{X}] \rightarrow [\mathcal{X}]$

\onslide<4>
for all $\mathcal{X} : X \Leftrightarrow \acute{X}$\\
\verb?  ?for all $(xs, \acute{xs}) \in [X]$\\
\verb?    ?$(g_A\ xs, g_{\acute{A}}\ \acute{xs}) \in [X]$

\onslide<5>
for all $f : X \rightarrow \acute{X}$\\
\verb?  ?for all $xs, \acute{xs} \in [X]$\\
\verb?    ?If $map\ f\ xs = \acute{xs}$\\
\verb?      ?Then $map\ f\ (g_A\ xs) = g_{\acute{A}}\ \acute{xs}$

\onslide<6>
for all $f : X \rightarrow \acute{X}$\\
\verb?  ?for all $xs \in [X]$\\
\verb?    ?$map\ f\ (g_A\ xs) = g_{\acute{A}}\ (map\ f\ xs)$\\
\verb? ?\\
$map\ f \circ g_A = g_{\acute{A}} \circ map\ f$

\end{overprint}

\end{frame}

\begin{frame}[fragile]
    \frametitle{Fold law for free}
\begin{itemize}
    \item \verb?foldr :: (X -> Y -> Y) -> Y -> [X] -> Y?\\
    \item If\\ 
    \verb?  f (g x y) = h (e x) (f y)?\\
    Then \\
    \verb?  f . foldr g u = foldr h (f u) . map e?
    \item Same as what we had before when e = id
\end{itemize}
\end{frame}

\begin{frame}[fragile]
    \frametitle{Other cool stuff}
\begin{itemize}
    \item Can work with predicates
        \begin{itemize}
            \item \verb?sort :: (X -> X -> Bool) -> [X] -> [X]?\\
            \item If\\
            \verb?  (x < y) = (a x < a y)?\\
            Then \\
            \verb?  map a . sort (<) = sort (<) . map a?
        \end{itemize}
    \item Can work with polymorphic equality
        \begin{itemize}
            \item So \verb?Eq a => ...? typeclass constraints are allowed
        \end{itemize}
\end{itemize}
\end{frame}

\section*{}

\begin{frame}
    \frametitle{Conclusion}
    \begin{itemize}
        \item Laws can arise without typeclasses   
        \item They can be handy
        \begin{itemize}
            \item Handy to use
            \item Handy to know
        \end{itemize}
        \item Polymorphic types correspond to free theorems
        \item Free theorems correspond to natural transformations
    \end{itemize}
\end{frame}

\begin{frame}
    \frametitle{Sources of things}
    \begin{itemize}
        \item Theorems for Free by Philip Wadler
        \begin{itemize}
            \item Builds on Reynolds' abstraction theorem
        \end{itemize}
        \item Algebra of Programming by Bird and De Moor has more advanced stuff 
        \item Slides at https://github.com/dalaing/bfpg-2013-06
    \end{itemize}
\end{frame}

\end{document}
